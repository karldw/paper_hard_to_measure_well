\begin{threeparttable}
 \caption{Expected fee, as a percentage of \(\delta\), with a 1\% annual audit budget and \(T =\) 3~months}
 \label{tab:expected-fee-1pct-3month}
\begin{tabularx}{\textwidth}{l d d d d}
\toprule
& \textcol{Mean (\%)}  &
  \textcol{Median (\%)}  &
  \textcol{p10 (\%)}  &
  \textcol{p90 (\%)} \\

\midrule
\multicolumn{5}{l}{Panel A: High fee \((\tau = 2 \delta)\)}\\
\midrule
\input{../tex_fragments/expected_fee_frac=1pct_tauT=high-3month.tex}\\

\midrule
\multicolumn{5}{l}{Panel B: Medium fee \((\tau = \delta)\)}\\

\midrule

\input{../tex_fragments/expected_fee_frac=1pct_tauT=med-3month.tex}\\

\midrule
\multicolumn{5}{l}{Panel C: Low fee \((\tau = \) \textdollar 5 per ton \ce{CO2e})}\\
\midrule
\input{../tex_fragments/expected_fee_frac=1pct_tauT=low-3month.tex}\\

\bottomrule
\end{tabularx}
\begin{tablenotes}
\item
\textsc{Note:}
Values are the expected fee per kg emitted, as a percentage of the social cost of emissions (100 \(T \tau r_i / \delta H\)).
Panels A, B, and C set \(T = \) 1 week and consider different values of \(\tau\).
Each row considers different audit rules to optimally allocate \(r_i\) according to the fixed audit budget, which is set to 1\% of all well pads.
Columns provide distributional statistics across well pads.
\(\delta = \) \$2 per kg methane.


Wells in this table are the sample of wells included in the \gls{AVIRIS-NG} sample (table~\ref{tab:well-summary-stats} panel~\textsc{a}).
Point estimates and square brackets indicate the mean and 95\% \gls{CI}.
(See text for \gls{CI} details.)
\end{tablenotes}
\end{threeparttable}
