\begin{threeparttable}
 \caption{Shortwave infrared (SWIR) remote sensors for observing methane point sources}
 \label{tab:new-satellite-platforms}
\begin{tabular}{l r r r r r}
\toprule
Instrument & 
Pixel  & 
SWIR spectral & 
Spectral  & 
Signal-to- & 
Observing  \\
 & 
size &
range  & 
resolution &
noise ratio &
record % Observing
\\
&
(km\textsuperscript{2}) &
(nm) &
(nm) &
(SNR) &
\\
\midrule
\multicolumn{6}{l}{Aircraft}\\
\midrule
AVIRIS-NG & 0.003 $\times$ 0.003 & 
1600--1700, 2200--2510 & 
5.0 & 200--400 & Campaigns\\
\midrule
\multicolumn{6}{l}{Satellite atmospheric sensors}\\
\midrule
SCIAMACHY & 30 $\times$ 60 & 1630--1670 & 1.4 & 1500 & 2002--2012\\
GOSAT & 10 $\times$ 10 & 1630--1700 & 0.06 & 300 & 2009--present\\
GHGSat & 0.05 $\times$ 0.05 & 1600--1700 & 0.3--0.7 & N/A & 2016--present \\
TROPOMI & 7 $\times$ 7 & 2305--2385 & 0.25 & 100 & 2017--present \\
AMPS & 0.03 $\times$ 0.03 & 1990--2420 & 1.0 & 200--400 & Concept\\
\midrule
\multicolumn{6}{l}{Satellite imaging spectrometers}\\
\midrule
PRISMA & 0.03 $\times$ 0.03 & 1600--1700, 2200--2500 & 10 & 180 & 2019--present \\
EnMAP & 0.03 $\times$ 0.03 & 1600--1700, 2200--2450 & 10 & 180 & 2020 \\
EMIT & 0.06 $\times$ 0.06 & 1600--1700, 2200--2510 & 7--10 & 200-300 & 2025 \\
SBG & 0.03 $\times$ 0.03 & 1600--1700, 2200--2510 & 7--10 & 200-300 & 2025 \\
CHIME & 0.03 $\times$ 0.03 & 1600--1700, 2200--2510 & <10 & In & 2025 \\
 &  &  &  &  preparation &  \\



\bottomrule
\end{tabular}
\begin{tablenotes}
\item \textsc{Note:} Table copied from \textcite{Cusworth/Jacob/Varon/Miller/Liu/Chance/Thorpe/Duren/Miller/Thompson/Frankenberg/Guanter/Randles:2019} table~1 (CC BY 4.0).
% Results simulate methane retrievals from the EnMAP satellite, expected to launch in 2021.
% Values in parentheses are standard deviations from five iterations.
% We exclude the paper's results for images with ``dark'' or ``urban'' backgrounds, as these include water and confuse the image processing algorithm.
% In personal communication, the lead author notes ``one could dramatically improve the prediction if there were some sort of decision tree that was based on the underlying surface.''



\end{tablenotes}
\end{threeparttable}
