\begin{threeparttable}
 \caption{Policy outcomes: Percent improvement from no-policy baseline\newline
(Audit budget = 1\% per year, \(T = \) 1~week)}
 \label{tab:policy-outcomes-1pct-1week}
\begin{tabularx}{0.85\textwidth}{l d d d}
\toprule

& \tau = 2\delta & \tau= \delta & \tau = \$5\\

\midrule
\multicolumn{4}{l}{A: \gls{DWL} improvement (\%)}\\
\midrule
\input{../tex_fragments/outcome_dwl_frac=1pct_tauT=all-1week.tex}\\

\midrule
\multicolumn{4}{l}{B: \(\mathbb{E}[\text{emiss}]\) improvement (\%)}\\
\midrule
\input{../tex_fragments/outcome_emis_frac=1pct_tauT=all-1week.tex}\\
\bottomrule
\end{tabularx}
\begin{tablenotes}

\item
\textsc{Note:}
Panels A and B show results for \gls{DWL} and emissions, both on a scale from 0 to 100,
where 0 is the no-policy baseline and 100 is the outcome of the infeasible first-best Pigouvian tax (higher is better).
Columns show different policy stringency levels \(\tau =\{2\delta, \delta, \$5\}\).
Rows are different constrained policy options, listed previously.
\gls{DWL} numbers include the costs of auditing.
Wells in this table are the sample of wells included in the \gls{AVIRIS-NG} sample.
Square brackets indicate 95\% CI.

\end{tablenotes}
\end{threeparttable}
