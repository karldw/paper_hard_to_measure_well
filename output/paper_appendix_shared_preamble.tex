

\documentclass[12pt,oneside,letterpaper]{article}

\nonstopmode

% Set a better default text area (note: don't use this and geometry at the same time.)
% \usepackage[DIV=10]{typearea}
\usepackage[margin=1.5in]{geometry}
\usepackage{fontspec}
\defaultfontfeatures{Ligatures=TeX}
\usepackage{amsmath}  % must be loaded before unicode-math
\usepackage{amsthm}
% Libertinus code suggested by https://tex.stackexchange.com/a/364502
\usepackage[
    math-style=ISO,
    bold-style=ISO,
    partial=upright,
    nabla=upright
]{unicode-math}
\setmainfont[Numbers={OldStyle, Proportional}]{Libertinus Serif}
\setsansfont{Libertinus Sans}
\setmathfont{Libertinus Math}
\newfontface\titlefont{Libertinus Serif Display}[Numbers={OldStyle, Proportional}, Fractions=Off,Ligatures=Common]
\newfontface\textfracfont{Libertinus Serif}[Numbers={OldStyle, Proportional}, Fractions=On]
\newfontface\tabularoldstylefont{Libertinus Serif}[Numbers={OldStyle, Monospaced}, Fractions=Off]

\usepackage{csquotes}
\usepackage[final]{microtype}
\frenchspacing{}
\microtypecontext{spacing=french}

\usepackage{ragged2e}  % allow for ragged-right with hyphenation

% Format sections:
% \titleformat{command}[shape]{format}{label}{sep}{before-code}[after-code]
\usepackage[it, small]{titlesec}

% Format itemized lists
\usepackage{enumitem}
\setlist{nolistsep} % more aggressive than noitemsep

% allow page breaks in math equations
\allowdisplaybreaks

\newcounter{axiomcounter}
\newcounter{theoremcounter}
\newcounter{examplecounter}
\newcounter{propositioncounter}
\newcounter{lemmacounter}
\newcounter{corollarycounter}

% Make it easier to change the spacing around these math environments
\newlength{\premathenv}
\setlength{\premathenv}{0.2\baselineskip}
\newlength{\withinmathenv}
\setlength{\withinmathenv}{0\baselineskip}
\newlength{\postmathenv}
\setlength{\postmathenv}{\premathenv} % default to the same, but could change.


\newcommand{\norm}[1]{\left\lVert#1\right\rVert}
\newcommand{\length}[1]{\left\lvert#1\right\rvert}

\theoremstyle{definition}
\newenvironment{theorem}[1]{%
\vspace{\premathenv}%
\refstepcounter{theoremcounter}%
\noindent\textbf{Theorem \thetheoremcounter}
{#1}
\vspace{\withinmathenv}
}{% end of beginning of environment, beginning of end of env
\vspace{\postmathenv}%
}
\newenvironment{proposition}[1]{%
\vspace{\premathenv}%
\refstepcounter{propositioncounter}%
\noindent\textbf{Proposition \thepropositioncounter}
{#1}
\vspace{\withinmathenv}
}{% end of beginning of environment, beginning of end of env
\vspace{\postmathenv}%%
}
\newenvironment{lemma}[1]{%
\vspace{\premathenv}%
\refstepcounter{lemmacounter}%
\noindent\textbf{Lemma \thelemmacounter}
{#1}
\vspace{\withinmathenv}
}{% end of beginning of environment, beginning of end of env
\vspace{\postmathenv}%
}
\newenvironment{corollary}[1]{%
\vspace{\premathenv}%
\refstepcounter{corollarycounter}%
\noindent\textbf{Corollary \thecorollarycounter}
{#1}
\vspace{\withinmathenv}
}{% end of beginning of environment, beginning of end of env
\vspace{\postmathenv}%
}
\newenvironment{axiom}[1]{%
\vspace{\premathenv}%
\refstepcounter{axiomcounter}%
\noindent\textbf{Axiom \theaxiomcounter}
{#1}
\vspace{\withinmathenv}
}{% end of beginning of environment, beginning of end of env
\vspace{\postmathenv}%
}
\newenvironment{example}[1]{%
\vspace{\premathenv}%
\refstepcounter{examplecounter}%
\noindent\textbf{EXAMPLE \theexamplecounter}
{#1}
\vspace{\withinmathenv}
}{% end of beginning of environment, beginning of end of env
\vspace{\postmathenv}%
}

\renewenvironment{proof}{%
\vspace{\premathenv}% previously \vspace{1\baselineskip}
\noindent{\textit{Proof:} }
}{% end of beginning of environment, beginning of end of env
\qed
\vspace{\postmathenv}% previously \vspace{1\baselineskip}
}
\newtheorem{case}{Case}
\renewcommand*{\thecase}{\Alph{case}}
\newtheorem{assumption}{Assumption}
\renewcommand*{\theassumption}{\Alph{assumption}}

\def\equationautorefname~#1\null{(#1)\null}

\newcounter{auditpolicy}
\setcounter{auditpolicy}{-1}
\newenvironment{auditpolicy}[1]{%
\refstepcounter{auditpolicy}%
{\tabularoldstylefont(\theauditpolicy)}~#1%
}


\usepackage{fancyhdr}
\pagestyle{fancy}
\lhead{}
\chead{}
\rhead{}
\lfoot{}
\cfoot{}
\rfoot{\thepage}
\renewcommand{\headrulewidth}{0pt}
\usepackage{xcolor}

% Redefine \left and \right with better spacing
\usepackage{mleftright}
\mleftright{}

% Define \expected and \indicator commands
\DeclareMathOperator{\blackboardE}{\mathbb{E}}
\newcommand{\expected}[1]{\blackboardE\left[#1\right]}
\newcommand{\indicator}[1]{\symbf{1}\left[#1\right]}



\usepackage{rotating} % provide the sidewaysfigure environment

% Set up tables
\usepackage[online,flushleft]{threeparttable}
\usepackage{dcolumn}
\newcolumntype{d}{D{@}{\hspace{0pt}}{4}}
\newcommand{\textcol}[1]{\multicolumn{1}{c}{#1}}

% \usepackage{sparklines}
\usepackage{tabularx}
\usepackage{longtable}
\usepackage{booktabs}

% Bibliography
\usepackage[
  authordate,
  isbn=false,
  backend=bibtex,
  autopunct=true,
  uniquename=false
  % backref=true
]{biblatex-chicago}
\renewcommand*{\bibfont}{\small}
\addbibresource{refs.bib}
\addbibresource{data_cites.bib}
\addbibresource{methane_measurement_refs.bib}

% use the \ce{} command for chemistry expressions like \ce{CO2}
\usepackage[version=4]{mhchem}

\usepackage{nameref}

% Set up a boolean so we can easily turn endfloat on and off.
% To turn on, change to \togglefalse{endfloat}
\newtoggle{endfloat}
\togglefalse{endfloat}
\iftoggle{endfloat}{%
  \usepackage{endfloat}
  % stopendfloat from putting everything on separate pages
  \renewcommand{\efloatseparator}{\mbox{}}
}{ % else :
  \usepackage{placeins} % provide \FloatBarrier command
}

% Define a boolean for whether we'll use the xr package.
\newtoggle{usexr}

\usepackage{graphicx}
\graphicspath{{../graphics/}}
\usepackage{import} % better relative import paths
\usepackage{tikz}
\usetikzlibrary{calc}
\usepackage{wrapfig} % provides wrapfigure and wraptable environments


% Define a command \blfootnote (blind foodnote or blank footnote)
% Redefine \thefootnote to remove the symbol
% Redefine \@makefntext to try to mimic the biblatex-chicago option footmarkoff, but only for the scope of this command.
% Then reset the counter.
\makeatletter
\newcommand{\blfootnote}[1]{%
  \begingroup
  \renewcommand{\thefootnote}{}%
  \renewcommand{\@makefntext}{}%
  \footnote{#1}%
  \addtocounter{footnote}{-1}%
  \endgroup
}
\makeatother


% Add an orcid logo:
% Borrowed partly from https://tex.stackexchange.com/a/445583
% And downloaded graphic from
% https://figshare.com/articles/ORCID_iD_icon_graphics/5008697
\usepackage{scalerel}
\newcommand\orcidicon[1]{\href{https://orcid.org/#1}{\mbox{\scalerel*{
\includegraphics{ORCIDiD_iconvector.pdf}
}{|}}}}


% Run texcount shell command and input result. Requires shell escape (or restricted shell escape) and associated security warnings
% The -merge option includes counts for any \input{}-ed files.
%\usepackage{shellesc}
%\newcommand{\wordcount}[1]{%
%\input|"texcount -1 -sum -merge -q #1.tex"%
%}


\usepackage{setspace}
\onehalfspacing
\usepackage{footmisc} % must be loaded before hyperref; allows references to footnotes.

\usepackage{url} % better URL line breaks.
% Ensure more achievable output (font embedding etc.) but see the docs about
% what it takes to be PDF/A-3b compliant.
% \usepackage[a-3b]{pdfx}



% Make hyperref happy. https://tex.stackexchange.com/a/256677
\providecommand*\propositioncounterautorefname{proposition}

\widowpenalty 9999
\predisplaypenalty=0 % lower than default 10000


%----Helper code for dealing with external references----
% (by cyberSingularity at http://tex.stackexchange.com/a/69832/226)
% https://www.overleaf.com/learn/how-to/Cross_referencing_with_the_xr_package_in_Overleaf

\makeatletter

\newcommand*{\addFileDependency}[1]{% argument=file name and extension
\typeout{(#1)}% latexmk will find this if $recorder=0
% however, in that case, it will ignore #1 if it is a .aux or
% .pdf file etc and it exists! If it doesn't exist, it will appear
% in the list of dependents regardless)
%
% Write the following if you want it to appear in \listfiles
% --- although not really necessary and latexmk doesn't use this
%
\@addtofilelist{#1}
%
% latexmk will find this message if #1 doesn't exist (yet)
\IfFileExists{#1}{}{\typeout{No file #1.}}
}\makeatother

\newcommand*{\myexternaldocument}[1]{%
\externaldocument{#1}%
\addFileDependency{#1.tex}%
\addFileDependency{#1.aux}%
}
%------------End of helper code--------------

